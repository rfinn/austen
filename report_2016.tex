\documentclass[12pt]{article}

\usepackage[vmargin=1in,hmargin=1in]{geometry}

\begin{document}

\begin{center}
{\Large
Reponse to Austen Group Report \\ 

\medskip
written by Rose A. Finn \\ \medskip  on behalf of the\\
\medskip
Department of Physics \& Astronomy \\ }
\end{center}


\section{Summary of Evidence}

We have invested a significant amount of time trying to understand how the quantities  in the Austen Group report were 
calculated.  In the end, we decided that it was easier to calculate cost, revenue and margin ourselves so that (1) we understand
how the quantities are calculated, and (2) we can create scenarios where we adjust input parameters and see how 
our margin is affected.   For reasons including confidentiality of salary information, our model makes several simplifying 
assumptions:
\begin{itemize}
\item all faculty at the same rank are assigned the same salary (\$70,000; \$80,000; \$95,000);
\item the overhead rate associated with faculty salaries is 40\%;
\item all sos students have the same discount rate of 65\%;
\end{itemize}


It is instructive to look at metrics while holding all salaries at a constant value of \$80,000/yr.  Revenue per SCH is same for all 
departments.  Variation in cost per sch among departments is then impacted by fraction of adjuncts (because no overhead associated w/adjuncts), department budget, and enrollment (b/c student credit hrs is in denominator).



The Austen Report

Department Profile
Other Measures

\section{Issues}
Examination

i. Careful analysis of data to identify specific areas of weakness.

ii. Depending on analysis, what options are available to address problem

iii. What is the likelihood of success of these options?

iv. What resources would be required to address them?

areas?


What students do we get, and where are they going - we get lower qualification students, yet they are going to good graduate schools and getting good jobs.  

What will happen to labs?  Faculty are teaching labs, will move to.  We should not get defensive about labs.  

SOS faculty is dramatically more effective than other schools than 3/3 load.  We should document what we do with reassigned time - independent studies, 

What is our measure of efficiency?

Internal demand - number of graduates

External demand - how many inquiries into major

We all need more space.  


\section{Growth Opportunities}

A large number of prospective students are interested in engineering.  While we currently 
have programs that lead to engineering (3/2 program; 4/1 with Union Graduate College/Clarkson),
we don't have our own engineering program.  This is clearly an opportunity for growth.  The question is
how to tap into some of this interest while increasing our margin.  A potential first step is to develop
an applied physics major.  Such a program prepares students as general engineers.  Students who complete
this prgram could go on to graduate school in engineering or into the a technical/engineering job. 

We are currently looking at existing programs in applied physics to see if we can reproduce the 
major here.

i. Is this revision of existing or new programs?

ii. What resources would be needed?

iii. What tasks are required and who should be involved in doing them?

Opportunities for growth, and what do we need to grow.
Applied physics
We could do these
NYU http://engineering.nyu.edu/academics/programs/applied-physics-bs
Oklahoma State http://physics.okstate.edu/www/degree-bs-applied.html
Georgia Tech http://www.catalog.gatech.edu/colleges/cos/physics/ugrad/phyapp/geninfo.php
Santa Cruz http://undergrad.pbsci.ucsc.edu/physics/apphbs/apphbs-requirements.html
Columbia University http://apam.columbia.edu/applied-physics-undergraduate-program
Hofstra http://bulletin.hofstra.edu/preview_program.php?catoid=80&poid=10321
Williams 
We would have trouble doing these
Illinois Institute of Technology https://science.iit.edu/programs/undergraduate/bachelor-science-applied-physics
NJIT http://physics.njit.edu/academics/undergraduate/bs-appliedphysics.php
Can’t access program description
Add 
Electronics II
Strength of materials
Junior level lab (3 credits per semester)
Senior design project
Optics
Fluids

\section{Possible Remedies to Problems}
Functioning well

i. What will be needed to sustain current strength?

More space


need to have a min of 12 students per course.  Classes with lower enrollment will be offerred every other year.

increase number of adjuncts


\section{Timetable for Implementing Program Changes}





\end{document}