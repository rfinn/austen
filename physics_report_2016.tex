\documentclass[12pt]{article}

\usepackage[vmargin=1in,hmargin=1in]{geometry}
\usepackage[colorlinks = true,
            linkcolor = blue,
            urlcolor  = blue,
            citecolor = blue,
            anchorcolor = blue]{hyperref}
\setlength{\parindent}{0in}
\setlength{\parskip}{0.1in}
\begin{document}

\begin{center}
{\large
Analysis of Program Efficiency\\ 
 Rose A. Finn, %  on behalf of the\\
Department of Physics \& Astronomy \\ }
\end{center}


\section{Summary of Evidence \& Issues}
\vspace*{-.2in}
\subsection{Evidence based on Department Profile}
{\bf STRENGTHS}

The Department of Physics and Astronomy is experiencing growth in the number of majors and minors.  The number of majors has increased from around 60 in Fall 2013 to 90 in Fall 2015, and the number of minors has increased from 5 to 20 over the same period.  The number of graduates in Spring 2016 was 18, and we expect this number to stay around 20 over the next few years.

We attract strong students.  Physics majors have a higher GPA than the college average. Students who enter Siena as physics majors have retention rates (including those who switch to a different major) that are higher than the average for the college.  

The size of our lab sections are larger than the average for the School of Science.  This helps to offset the cost of running labs.  While we don't have the same safety issues as chemistry, we need continued support for learning assistants in lab and lecture sections to ensure safety and efficacy of larger section sizes.

{\bf WEAKNESSES}

Our average class size is less than that of the college as a whole and comparable to the average for the School of Science.   The size of our upper-level classes are lower than both the college and SOS averages.  We have three solutions for this.  First, as the larger cohorts of current freshmen and sophomores make their way up through the program, the upper-level class sizes will naturally increase.  Second, we will offer courses that have consistently had low enrollment every other year instead of every year.  Third, we will work to increase the number of students who pursue minors in physics and astronomy.  

We need to look at the impact that our engineering pathways have on the enrollment in upper-level courses.  By design, students in the 3/2 program complete their 4th year at a cooperating engineering school, and they transfer back their credits to Siena to complete their Siena graduation requirements by the fourth year.  Typically they will transfer upper-level physics courses, and this results in lower enrollment in our upper level physics courses.  Similarly, students in the 4/1 program take three course at Union while they are still Siena students, and these are frequently counted as an upper-level physics course.  This again means lower enrollment for our physics courses.  The engineering pathways bring in large numbers of students, so we obviously want to make the programs work.  However, we need to look at how this impacts our enrollment numbers.  This is an example of how the artificial distinction between lower and upper level courses works against our program.  We should be able to offset lower enrollment upper-level courses with high enrollment lower-level courses.

According to our department profile, our 4-year graduation rates are startlingly low.  I need to understand how these are calculated so that we can address the issue.  If this is calculated as the number of graduates divided by the number of entering students in a given year, then our low rate is a consequence of growth where incoming class sizes are exceeding graduating class sizes. In our particular case, a low graduation rate might actually be a positive.

\subsection{Evidence based on Austen Report}
We have invested a significant amount of time trying to understand how the quantities in the Austen Group report were calculated.  In the end, we decided that it was easier to calculate cost, revenue, and margin ourselves so that (1) we understand how the quantities are calculated, and (2) we can create scenarios where we adjust input parameters and see how  our margin is affected.   For reasons including confidentiality of salary information, our calculations makes several simplifying assumptions:
\begin{itemize}
\item all faculty at the same rank are assigned the same salary (\$70,000; \$80,000; \$95,000);
\item the overhead rate associated with faculty salaries is 40\%;
\item all School of Science students have the same discount rate of 65\%;
\end{itemize}

{\bf Results:  }
According to our calculations, our margin per student-credit-hour (SCH) is highest within school of science.  This is different from the results of the Austen report, which ranks physics below computer science and environmental science in terms of margin per student credit hour.  We are working to understand the remaining discrepancies between our numbers and those of the Austen report.  One potential explanation is that our model assumes the same discount rate for all school of science students.  If physics students have a higher discount rate than computer science and environmental science students, then this could account for the difference between our estimate of margin and that of the austen report.  This also might indicate that physics students are stronger applicants, which of course is would be positive.  


When looking at upper-level classes only (course numbers greater than 200), our department is running at a deficit.  We feel the overall margin in a better metric as the lower/upper level division is arbitrary and seems to undervalue the time and effort that we put into offering quality, engaging core classes.  Nevertheless, we expect the enrollment in upper-level classes to increase over the next few years as described above.  This will improve both the overall and major margins.  

Our calculated average class size of 20.1 is highest within the SOS over the two-year period covered by the Austen report.  This is close the average of 21 listed in the Austen report.

%It is instructive to look at metrics while holding all salaries at a constant value of \$80,000/yr.  %Revenue per SCH is same for all 
%departments.  Variation in cost per sch among departments is then impacted by fraction of adjuncts %(because no overhead associated w/adjuncts), department budget, and enrollment (b/c student credit %hrs is in denominator).


\subsection{Other Measures}

The Physics faculty run active research programs that are supported by external funding agencies and results in numerous publications and student presentations.  In addition, these grants brings resources to the college that directly benefit our students.  As a result of extensive research opportunities and a strong academic program, our graduates are successful in obtaining employment and position in graduates schools.  We argue that external grants should be including when analyzing margins.


% \section{Issues}
% Examination

% i. Careful analysis of data to identify specific areas of weakness.

% ii. Depending on analysis, what options are available to address problem

% iii. What is the likelihood of success of these options?

% iv. What resources would be required to address them?

% areas?


% What students do we get, and where are they going - we get lower qualification students, yet they are going to good graduate schools and getting good jobs.  

% What will happen to labs?  Faculty are teaching labs, will move to.  We should not get defensive about labs.  

% SOS faculty is dramatically more effective than other schools than 3/3 load.  We should document what we do with reassigned time - independent studies, 

% What is our measure of efficiency?

% Internal demand - number of graduates

% External demand - how many inquiries into major

% We all need more space.  

\section{Possible Remedies to Problems}
\vspace*{-.2in}
{\bf Increase overall course enrollment:}
While the goal is obvious, the best method for implementing it is not.  We need to have the option of running some low-enrollment classes.  For example, the rising sophomore class is large, and the number of students exceeded our maximum enrollment of 30 in the sophomore-level modern physics class.  The class is a pre-requisite for other physics classes, so if we tell students who weren't able to register that they need to wait until next year, this will negatively impact their ability to graduate in four years.  Clearly, this is unfair and even unethical.  Instead, we opened an additional section of modern physics to accomodate the ten additional students, which results in a low-enrollment course.  We expect similar issues to arise while we are in a period of growth.  

While a minimum enrollment policy does not make sense, we can take steps to increase overall enrollment.  For example, 
upper-level classes with lower enrollment will be offerred every other year instead of evey year.  In addition, as the larger underclass cohorts move up, our class sizes will increase as long as we are able to maintain enrollments of 20-30 in the incoming classes.

{\bf Increase use of adjunct faculty:} 
In addition to increasing enrollment, another way to increase our margin is to increase the fraction of adjunct faculty.  The general physics labs are particularly well-suited for adjuncts because we can offer them once a week in the evening.  Moving to more adjuncts will require us to use a general physics coordinator again to faciliatate communication with the adjuncts and to make sure all courses and lab sections are in sync.  We should also consider hiring professional engineers as adjuncts as a way to increase our engineering course offerings.  We strongly encourage the college to increase the pay for adjunct faculty so that this becomes a more attractive option for high-school physics teachers and working engineers.  The savings of using adjunct faculty is significant soley due to the fact that there is no associated overhead for healthcare and other benefits.

{\bf Improved marketing:}  According to several metrics, our program attracts strong students to Siena.  However, these students seem to have a relatively high discount rate, meaning that Siena had to offer the students more money before they agreed to attend.  We believe that our program is strong and should attract strong students without having to increase the discount rate.  A major missing ingredient in our current recruitment efforts is effective marketing.  We need to make our department website informative and attractive, and we need to work closely with the new marketing department to increase awareness of our program.  

\newpage
\section{Growth Opportunities}
\vspace*{-.2in}
A large number of prospective students are interested in engineering.  While we currently 
have programs that lead to engineering (3/2 program; 4/1 with Union Graduate College/Clarkson),
we don't have our own engineering program.  This is clearly an opportunity for growth.  The question is
how to tap into some of this interest while increasing our margin.  A potential first step is to develop
an applied physics major.  Such a program prepares students as general engineers who can move directly into the workforce or who can continue their education at a graduate engineering school.

We are currently looking at existing programs in applied physics to see if we can reproduce the 
major here.  The following is a list of applied physics programs that would be straight-forward to emulate at Siena:
\vspace*{-.4cm}
\begin{itemize}
\item \href{http://engineering.nyu.edu/academics/programs/applied-physics-bs}{NYU},  
\vspace*{-.2cm}\item \href{http://physics.okstate.edu/www/degree-bs-applied.html}{Oklahoma State},  
\vspace*{-.2cm}\item \href{http://www.catalog.gatech.edu/colleges/cos/physics/ugrad/phyapp/geninfo.php}{Georgia Tech}, 
\vspace*{-.2cm}\item \href{http://undergrad.pbsci.ucsc.edu/physics/apphbs/apphbs-requirements.html}{Santa Cruz},
\vspace*{-.2cm}\item \href{http://apam.columbia.edu/applied-physics-undergraduate-program}{Columbia University},
\vspace*{-.2cm}\item \href{http://bulletin.hofstra.edu/preview\_program.php?catoid=80\&poid=10321}{Hofstra}.
\end{itemize}

Courses that we should consider adding include:  
\vspace*{-.4cm}
\begin{itemize}
\item electronics II, 
\vspace*{-.2cm}\item  strength of materials,
\vspace*{-.2cm}\item  a junior-level lab (3 credits per semester),
\vspace*{-.2cm}\item  a senior design project instead of the advanced lab course,
\vspace*{-.2cm}\item  optics, and 
\vspace*{-.2cm}\item  fluids.
\end{itemize}
We need to look carefully at staffing issues to see if we can cover these and current course with existing faculty or if additional resources are necessary.  In addition, our department has reached capacity regarding office and lab space, so we need to consider if we can physically accomodate such a new program.

We will establish an applied physics working group that will meet over the next year.  They will:
\vspace*{-.4cm}
\begin{itemize}
\vspace*{-.2cm}\item  develop a plan for an applied physics major; 
\vspace*{-.2cm}\item  identify its strengths, weaknesses, and resources (space, staffing, budget) needed to bring it to fruition;
\vspace*{-.2cm}\item  identify the enrollment that will be required to make the program viable.
\end{itemize}
% i. Is this revision of existing or new programs?

% ii. What resources would be needed?

% iii. What tasks are required and who should be involved in doing them?

% Opportunities for growth, and what do we need to grow.
% Applied physics



%We would have trouble doing these
%Illinois Institute of Technology %https://science.iit.edu/programs/undergraduate/bachelor-science-applied-physics
%NJIT http://physics.njit.edu/academics/undergraduate/bs-appliedphysics.php
%Can’t access program description




\newpage
\section{Timetable for Implementing Program Changes}

\begin{table}[h]
\begin{tabular}{|p{2.5in}|p{1.25in}|p{1.25in}|p{1in}|}
\hline
{\bf Action} & {\bf Implementation} & {\bf Expected ~~~ Result}  & {\bf Implemented by}\\
\hline \hline
 & & & \\ 
Move low enrollment upper-level courses to every other year. & Fall 2017 & Fall 2019 & DH\\ &  & &  \\ \hline
 & & & \\
Increase number of students pursuing physics and astronomy minors. & Fall 2017 & Fall 2019 & all faculty\\ &  & &  \\ \hline
 & & & \\ 
Review applied physics major & AY 2016-2017 & AY 2017-2018 & ad-hoc ~~~~ committee\\  & & & \\ \hline
 & &  &\\ 
Increase overall enrollment by 2 students per course &underway & Fall 2018 & all faculty \\ & & & \\ \hline

 & &  &\\ 
Increase fraction of courses taught by adjuncts. & Spring 2017 & Fall 2018 & DH \\
 & & & \\  \hline

& & & \\ 
Improve Physics website. & Summer 2016 & Fall 2017 & ad-hoc ~~~~ committee \\
 & & & \\  \hline

& &  & \\ Work closely with the new marketing department  to increase regional and national awareness of the physics program.& Summer 2016 & Fall 2017 & all faculty \\
 & & & \\  \hline
\hline
\end{tabular}
\end{table}




\end{document}